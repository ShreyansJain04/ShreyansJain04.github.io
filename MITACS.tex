\documentclass[a4paper,11pt]{cdsTemp}
%-----------------------------------------------------------
\usepackage[]{geometry}
% top=0.55in, left=0.55in, right=0.85in
% \geometry{
%   top=0.55in,
%   bottom=0.15in,
%   left=0.55in,
%   right=0.85in,
%   includeheadfoot,
%   headsep=0mm,
%   footskip=0mm
% }

%%%% MINE
\geometry{
  top=0.4in,
  bottom=0.25in,
  left=0.45in,
  right=0.45in,
  includeheadfoot,
  nofoot,
  nohead,
  headsep=0mm
}
  
  
%   %%% CDS
%   \geometry{
%   top=0.55in,
% %   bottom=0.1in,
%   left=0.55in,
%   right=0.85in,
% includeheadfoot,
%   nofoot,
%   nohead,
%      headsep=0mm
% %   ,showframe
%   }
\usepackage{graphicx}
\usepackage{url}
\usepackage{hyperref}


\usepackage{palatino}
\usepackage{booktabs}

\ifxetexorluatex
  % If using xelatex or lualatex:
  \setmainfont{Lato}
\else
  % If using pdflatex:
  \usepackage[default]{lato}
\fi
% \usepackage{fontawesome}

\usepackage{xcolor}
\fontfamily{SansSerif}
\usepackage{hyperref}
\usepackage{fancyhdr}

\usepackage[T1]{fontenc}
\usepackage
%[ansinew]
[utf8]
{inputenc}

\usepackage{color}
\definecolor{boxbg}{rgb}{0.5,0.5,0.5}

\raggedbottom

\setlength{\tabcolsep}{0in}
\newcommand{\isep}{-2 pt}
\newcommand{\lsep}{-0.5cm}
\newcommand{\psep}{-0.6cm}
\renewcommand{\labelitemii}{$\circ$}

\pagestyle{plain}
\textheight=11.2in

% Change the colours if you want to
\definecolor{VividPurple}{HTML}{1560BD}
\definecolor{SlateGrey}{HTML}{000000}
\definecolor{LightGrey}{HTML}{666666}
\definecolor{red}{HTML}{666666}

\colorlet{tagline}{VividPurple}
\colorlet{heading}{VividPurple}
\colorlet{headingrule}{VividPurple}
\colorlet{accent}{VividPurple}
\colorlet{emphasis}{SlateGrey}
\colorlet{body}{LightGrey}

\makeatletter
\newcommand{\globalcolor}[1]{%
  \color{#1}\global\let\default@color\current@color
}
\makeatother
\definecolor{main}{rgb}{0.0,0.05,0.05}
\definecolor{primary}{rgb}{0.0,0.16,0.27}

\AtBeginDocument{\globalcolor{main}}

\newcommand{\resitem}[1]{\item #1 \vspace{-2pt}}
\newcommand{\headingnote}[1]{{\normalfont \footnotesize \textit{ #1}}}
\newcommand{\resheading}[1]{{ \colorbox{boxbg}{\begin{minipage}{0.96\textwidth}{\textcolor{primary}{\textbf{#1 \vphantom{p\^{E}}}}}\end{minipage}}}}

\usepackage{textcomp}

\begin{document}

\huge{\textbf{Shreyans Jain}}\\
\\
[0.2cm]

\large{\color{red}{Third Year Undergraduate}}

\\[0.2cm]

\normalsize{\color{red}{Major in Artificial Intelligence  $\cdot$ Major in Electrical Engineering $\cdot$ Indian Institute of Technology Gandhinagar}}

\personalinfo{
    \email{\href{mailto:shreyans.jain@iitgn.ac.in}{shreyans.jain@iitgn.ac.in}}
    \linkedin{\underline{\href{https://www.linkedin.com/in/shreyans-jain-1b56aa247/i}{Shreyans Jain}}}
    \github{\underline{\href{https://www.github.com/shreyansjain04}{shreyansjain04}}}
}
\makecvheader
\cvsection{ACADEMIC QUALIFICATIONS}

\begin{tabular}{ p{1.7cm} @{\hskip 0.08in} p{4.254cm} @{\hskip 0.08in} p{8.054cm} @{\hskip 0.09in} p{2.554cm} @{\hskip 0.08in} p{1.72cm} } 
\textbf{Degree} & \textbf{Specialization} & \textbf{Institute} & \textbf{Year} & \textbf{CPI/\%} \\
\midrule
B.Tech. & \textit{Electrical Engineering} & IIT Gandhinagar & 2022-Present& 8.62/10\\
Class XII  & \textit{Physics, Chemistry, Maths}  &  Maa Bharti Sr.Sec.School,  Swami Vivekanand Nagar& 2021-2022& 96\%\\
Class X  &  & FIITJEE World School, Secunderabad& 2019-2020& 10/10\\

\end{tabular}


\cvsection{Publications}
\begin{itemize}
    \item Panda S., Varun M.S.$^{\ast}$, \textbf{Shreyans Jain}$^{\ast}$, Maharana, S. K., Prathosh, A. P. (2024). "\textbf{Variational Diffusion Unlearning: A Variational Inference Framework for Unlearning in Diffusion Models}" -\textit{Accepted at NeurIPS Workshop SafeGenAI and submitter further to AISTATS}. 
 \href{https://openreview.net/forum?id=B2wDjiED9V}{[Paper]} 
\end{itemize}

\begin{itemize}
    \item \textbf{Shreyans Jain}$^{\ast}$, Viraj Vekaria$^{\ast}$, Karan Gandhi$^{\ast}$, Aadya Arora$^{\ast}$ (2024). 
    "\textbf{WavShadow: Wavelet Based Shadow Segmentation and Removal}" 
    -\textit{Accepted at ICVGIP 2024}. 
    \href{https://doi.org/10.48550/arXiv.2411.05747}{[Paper]}
\end{itemize}


\begin{itemize}
    \item  \textbf{Shreyans Jain}, Mihir A.,  Ruchika M., Nishita M., Himanshu S. (2024). "\textbf{Suppressing Streak Artifacts in Ultrasound Images during Therapy Guidance by a Hybrid U-Net and Enhanced Masked Autoencoder}" - \textit{Accepted at Conference International Society for Therapeutic Ultrasound, Taipei 2024.} 
   
\end{itemize}

\begin{itemize} \itemsep \isep
    \item Mihir A., \textbf{ Shreyans Jain},  Ruchika M., Nishita M., Himanshu S. (2024). "\textbf{Suppressing Streak Artifacts Generated by the Interference of Imaging and Therapy Fields: Initial findings using a Hybrid U-Net and Diffusion Model}" -
    \textit{Accepted at Conference IEEE South Asian Ultrasonics Symposium 2024.} 
    \href{https://github.com/ShreyansJain04/Streak-Removal-in-HIFU-Images}{[Code Repository]} 
    \href{https://iitgnacin-my.sharepoint.com/:b:/g/personal/22110245_iitgn_ac_in/EWChVmzi6PpFj8X1gaOIiaoBzbb4Lco_DXAnMhD06wOKfA?e=Qf6hNT}{[Poster]}
\end{itemize}

\begin{itemize}
    \item Ruchika Dhawan, Mihir Agarwal$^*$, {\textbf{Shreyans Jain}}$^*$, {Hrriday Ruparel}$^*$,  Himanshu Shekhar\\
\textbf{Reconstruction of Ultrasound Super-Resolution
Images using a Hybrid Attention-based U-Net Architecture applied to sparse data} -
\textit{Accepted at Conference International Society for Therapeutic Ultrasound, Taipei 2024.} 

\end{itemize}
\vspace{-0.2cm}

\cvsection{Internships}
\cvevent{Research Project, IISc Bangalore | Unlearning Concepts from Diffusion Models}{Advisor \labelitemii  \hspace{0.5mm} Prof. Prathosh A.P \labelitemii\hspace{0.5mm} {Representation Learning Lab}, IISc }{\color{red}May '24 - Ongoing}{}
\begin{itemize}

    \item Developing a method for unlearning specific concepts from diffusion models  with minimal interference in unrelated concepts by creating an adapter network and defining a novel loss function to reduce  the likelihood of the concepts to be erased.
\end{itemize}
    
\divider

\vspace{0.1cm}
\cvevent{AI Engineering Intern, KrishiMandir}{Advisors \labelitemii  \hspace{1mm} Mr. Sumeet Mohanty, Mr. Noel Kurian}{\color{red}December '23 - May '24
}{}
\begin{itemize}
    \item Implementing, testing, and benchmarking various open-source Large Language Models (LLMs) and Retrieval-Augmented Generation (RAG) models to optimize Clojure function calling, enhancing both efficiency and accuracy.
    \item Developing Visual Similarity Search System achieving an accuracy of 96\%, aiming to replace current OCR mechanisms employed in the company.
    \end{itemize}
    


\cvsection{SELECTED PROJECTS}
\cvevent{AI Image Classification and Artifact Identification}
{Silver Medal Winner | Inter IIT Tech Meet 13.0 | Ms. Aishwarya Agarwal | Adobe Research | \href{https://github.com/ShreyansJain04/AI-Image-Detection-Artifact-Identification}{Project Link}}{Ongoing}{}
\begin{itemize}
    \item Initially developed a robust detection framework using the LeViT model and Neighboring Pixel Representation (NPR) to identify AI-generated images, achieving 99.21\% accuracy on the CIFAKE dataset and securing the highest accuracy among all competitors at the Inter IIT Tech Meet 13.0.
    \item Collaborating with Ms. Aishwarya Agarwal at Adobe Research to extend the project by designing techniques that are more robust and effective on out-of-distribution generative models.
    \item Exploring advanced approaches such as adversarial training, jigsaw-style self-supervised learning, and continual learning to improve the model's adaptability and resilience against novel generative artifacts.
    \item Aiming to enhance detection accuracy and generalization across diverse AI-generated content by leveraging innovative methodologies to handle out-of-distribution challenges.
\end{itemize}
\divider

\newpage

\cvevent{Streak Artifact Suppression in High-Intensity Focused Ultrasound and Histotripsy Images for Therapeutic Applications}{Research Project | Prof. Himanshu Shekhar | MUSE Lab, IIT Gandhinagar}{Ongoing}{}
\begin{itemize}
    \item Developed a Fourier-attention U-Net architecture to suppress streak artifacts in B-mode ultrasound images, facilitating clearer visualization for therapeutic ultrasound applications.
    \item Implemented synthetic streak generation for training data to simulate real-world artifact conditions, enhancing the model's robustness in detecting and inpainting streaks.
    \item Achieved a significant increase in Signal-to-Noise Ratio (SNR) by 20 dB across 154 test cases, with an Intersection over Union (IoU) score of 0.698, demonstrating rapid convergence and potential for real-time clinical application.
\end{itemize}


% \cvevent{Reconstruction of Ultrasound Super-Resolution Images using a Hybrid Attention-based U-Net Architecture applied to sparse data}{Research Project | Prof. Himanshu Shekhar | MUSE Lab, IIT Gandhinagar}{Ongoing}{ }

% \begin{itemize}
%     \item Developed a Hybrid Attention-Based U-Net architecture to enhance Ultrasound Localization Microscopy (ULM) by accurately tracking microbubbles in noisy image sequences, facilitating super-resolution imaging in therapeutic applications.
%     \item Reduced the data requirements for ULM by 40 times, utilizing sparse data to achieve high-quality super-resolution images, a significant improvement over traditional methods, thus making ULM feasible for clinical ultrasound applications.
%     \item Achieved peak signal-to-noise ratio (SNR) enhancement of 14.73 dB in simulated data, validating the model's capability to capture detailed vascular features with limited data input compared to conventional ULM reconstruction techniques.
% \end{itemize}
% \divider

\cvevent{Reconstruction of Ultrasound Super-Resolution Images using a Hybrid Attention-based U-Net Architecture applied to sparse data}
{Research Project | Prof. Himanshu Shekhar | MUSE Lab, IIT Gandhinagar}{Ongoing}{}
\begin{itemize}
    \item Developed a Hybrid Attention-Based U-Net architecture to enhance Ultrasound Localization Microscopy (ULM) by accurately tracking microbubbles in noisy image sequences, facilitating super-resolution imaging in therapeutic applications.
    \item Reduced the data requirements for ULM by 40 times, utilizing sparse data to achieve high-quality super-resolution images, a significant improvement over traditional methods, thus making ULM feasible for clinical ultrasound applications.
    \item Achieved peak signal-to-noise ratio (SNR) enhancement of 14.73 dB in simulated data, validating the model's capability to capture detailed vascular features with limited data input compared to conventional ULM reconstruction techniques.
\end{itemize}
\divider


\cvevent{Visualisations for Machine Learning}{Advisor: Prof. Nipun Batra | Sustainability Lab, IIT GN | \underline{\href{https://intensedrop.notion.site/Visualizations-for-ML-951994a20ee447268626a82e4659c41f}{Project Link}}}{\color{red}July '23 - November '23}{}
\begin{itemize}
    \item In collaboration with a team of three students, developed over 50 interactive machine learning educational tools like interactive blogs and Streamlit applications to visualize complex concepts and demystify the mathematics behind them.
    \item Focused on elucidating a range of topics, including optimization, Markov chains, Fisher information, Shannon-Fano coding, JPEG compression, and CORDIC algorithms broadly covering optimization, information theory, and probability.
\end{itemize}
\divider



\cvevent{Variational Autoencoder on FPGA}{Course Project | Prof. Joycee Mekie | IIT Gandhinagar | \href{https://github.com/ShreyansJain04/DigitalSystemsVAE}{Project Link}}{March '24 - April '24}{}
\begin{itemize}
    \item Design and Architecture: Developed a 16-bit fixed-point quantized VAE model for processing the MNIST dataset on a Nexys 4 FPGA.  Ensured that the network’s generative quality remained consistent with the software baseline, verifying image reconstruction fidelity on test samples.
    \item Verification and Testing: Simulated the complete hardware design using Verilog testbenches before deployment. Compared output distributions and reconstructed images against software-based PyTorch models. Confirmed that the FPGA implementation matched the software baseline within acceptable error margins.
    \item Performance and Results: Achieved a latency reduction and real-time reconstruction capability for MNIST digits, enabling near-instantaneous inference.
\end{itemize}

\divider

\cvevent{Bicycle Safety App for Android}{Advisor \labelitemii  \hspace{0.5mm} Prof. Nithin V George, Electrical Engineering \labelitemii\hspace{0.5mm} IIT GN }{\color{red}July '23 - November '23}{}
\begin{itemize}
 \item The app employs phone sensors for real-time detection of over-speeding, falls, and geographical boundary breaches. 
 \item Features safety measures, including automatic alarm notifications on both the child's and parents' devices in case of overspeeding, falling, or leaving a predefined geographic area. If the child doesn't confirm safety within 5 seconds of an alarm due to a fall, the system activates the child's phone microphone, allowing parents to listen in for response.         
\end{itemize}
\divider

\cvevent{Multimodal Content Analysis and Generation for Social Media Platforms\hfill\color{red}}{{Inter IIT Tech Meet 12.0 }}{December '23-January '24}{}
\begin{itemize}
 \item Developed a sophisticated DNN for predicting the popularity of tweets (likes) using multimodal representation of data including timestamps, content, and images using ResNet-50 and Universal Sentence Encoder. 
 \item Utilized BLIP for initial image-based captioning, enriching content with media insights, and then fine-tuned LLaMA-2 for enhanced tweet accuracy and relevance. This approach established a pipeline effectively combining image captioning and large language models to generate contextually rich and engaging tweets.
\end{itemize}
\divider



\cvevent{DC Anemometer}{Advisor \labelitemii  \hspace{0.5mm} Prof. Arup Lal Chakraborty, Electrical Engineering \labelitemii\hspace{0.5mm} IIT GN\labelitemii\hspace{0.5mm} \underline{\href{https://github.com/ShreyansJain04/DC_Anemometer}{Project Link}}}{\color{red}March '23 - April '23}{}
\begin{itemize}
    \item Developed an anemometer using a DC motor and Arduino UNO. This device operates on the principle that the generated voltage is directly proportional to the motor's rotational speed, enabling accurate wind speed measurement.
    \item Conducted Extensive Data Collection and cleaning and tested out various ML Algorithms and performed calibration in different test conditions.
\end{itemize}
\divider

\newpage



\cvsection{Technical Skills}
\textbf{Languages:} \cvtag{Python}\cvtag{C++}\cvtag{C}\cvtag{Bash}
\\\textbf{Tools:} \cvtag{MATLAB} \cvtag{\LaTeX}\cvtag{NanoHub}\cvtag{Simulink}\cvtag{Autodesk Inventor}\cvtag{Azure}\cvtag{AWS}\cvtag{MetaSploit}\cvtag{BurpSuite}
\\\textbf{Libraries:} \cvtag{OpenCV}\cvtag{Matplotlib}\cvtag{Numpy}\cvtag{Pandas}\cvtag{Git}\cvtag{Github}
\\\textbf{Frameworks:} \cvtag{Keras}\cvtag{Tensorflow}\cvtag{Pytorch}


\cvsection{ACHIEVEMENTS }
\begin{itemize}    
    \item Secured the \textbf{Silver Medal} at Inter IIT Tech Meet 13.0 in the Computer Vision Problem Statement by Adobe, competing against teams from all 23 IITs.
    \item Secured \textbf{AIR 3} across 40,000 candidates in IMU-CET 2022.
    \item Awarded a Research Consultant position and cash prize by \textbf{WorldQuant} for achieving a top rank in the IITGN Alphathon.
    \item \textbf{Dean’s List} award for excellent academic performance in Semesters I, II & IV
    \item \textbf{Secured 3rd position} in the CTF competition and \textbf{1st among first years} in the Machine Learning challenge
in the Annual College Hackathon - HackRush
    \item Selected for \textbf{NATIONAL SQUAD} in Sailing.
\end{itemize}



\cvsection{Relevant Courses}
\\[0.05cm]\textbf{Completed Institute Courses: }\textit{Computer Vision, Digital Signal Processing, Machine Learning, Control Systems, Digital Systems, Data Structures and Algorithms, Signals, Systems and Random Processes , Data-Centric-Computing, Probability Statistics and Data Visualisation, Ordinary Differential Equations, Linear Algebra, Calculus.}
\\[0.05cm]\textbf{Online Courses:} \textit{Certificate of Machine Learning courses from specialization by Deeplearning.ai on Coursera}

\cvsection{POSITIONS OF RESPONSIBILITY \& EXTRACURRICULARS} 

\begin{itemize}
    \item \textit{\textbf{Machine Learning Club Secretary} \hfill \color{red}May '23 - Ongoing}
    
    Organized and led weekly ML reading groups and hackathons, facilitated hands-on workshops on neural networks and optimization, and coordinated inter-college ML competitions. Mentored junior members to foster a strong ML community on campus, encouraging peer learning and research collaborations.
    
    \divider

    \item \textit{\textbf{Co-Licensee and Sponsorship Lead at TEDxIITGandhinagar}  \hfill \color{red}November '23 - Ongoing}

    Spearheaded TEDxIITGandhinagar, securing first-ever sponsorship in the event’s history and surpassing budget
goals. Led a 70+ member team to deliver a full-house event with 8 speakers, marking a milestone in the program’s
history.My role involved strategic planning, team coordination, and fostering a platform for thought-provoking discussions and ideas.

    \divider 

    \item \textit{\textbf{General Member of The Technical Council}  \hfill \color{red}May '23 - Ongoing}

    Involved in executing new initiatives, fostering external collaborations, and promoting the Council's role among students. My responsibilities included assisting technical clubs with project development and orchestrating various events, in a bid to improve the college's technical culture and student engagement in tech-related activities.

    \divider
	
    \item \textit{\textbf{Core Member-Systems programming group}  \hfill \color{red} July '23 - Ongoing}

    As a core member I spearheaded a CTF event with over 70 participants, I also administered a bWAPP environment, enhancing practical security learning. My role encompasses organizing and executing various cybersecurity competitions and events, contributing to community growth.

\end{itemize}

\end{document}
